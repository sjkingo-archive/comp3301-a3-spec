\documentclass[12pt,a4paper]{article}

%% fonts
\usepackage{fontspec}
\setromanfont[Mapping=tex-text]{Minion Pro}
\setmonofont{Inconsolata}
\usepackage[normalem]{ulem}

\usepackage[hmargin=2.5cm]{geometry}

%% flexible lists
\usepackage{paralist}

%% more sane tables
\usepackage{booktabs}
\usepackage{longtable}

%% graphics
\usepackage[xdvi]{graphicx}
\usepackage{rotating}

%% add a bit more spacing between paragraphs
\renewcommand{\baselinestretch}{1.1}

%% hyperlinks -- must be last
\usepackage{color}
\usepackage[colorlinks=true,linkcolor=blue,urlcolor=blue]{hyperref}


%% metadata for pdf 
\newcommand{\pdfsubj}{COMP3301\slash COMP7308}
\newcommand{\pdftitle}{Assignment 3 --- File systems}
\newcommand{\pdfauthor}{John Williams and Sam Kingston}
\newcommand{\assignnum}{a3}
\newcommand{\assignweight}{20\% }
\title{\pdftitle}
\author{\pdfauthor}
\hypersetup{
    pdftitle={\pdftitle},
    pdfauthor={\pdfauthor},
    pdfsubject={\pdfsubj}
}


\newcommand{\duedate}{8pm Thursday 27 October 2011}

%% end of preamble

\begin{document}

\begin{center}
\large \textbf{The University of Queensland} \\
\large \textbf{School of Information Technology and Electrical Engineering} \\
\large \textbf{Semester Two, 2011}
\end{center}

\begin{center}
\Large \textbf{\pdfsubj} \\
\Large \textbf{\pdftitle}
\end{center}

\begin{center}
\large \textbf{Due: \duedate} \\
\large \textbf{Weighting: \assignweight for COMP3301 students (100 marks total)}
\end{center}


\begin{center}
Version 1.0 --- 6 October 2011
\end{center}

\subsection*{Introduction}

The goal of this assignment is to give you practical experience modifying
an existing file system for Linux.

You will need to complete the file systems practical before attempting this
assignment, as it instructs you how to clone the \textit{ext3301} file system
that you will be using for the assignment.

As part of the assessment you must also answer some short--response questions
that will test your understanding of the concepts that you have implemented.

This assignment may be completed \textbf{individually} or in
\textbf{self-selected groups of two}. Please read the section on group work if
you decide to work in groups of two. Although you may share code with your team
member (if working in groups of two), it is still considered cheating to look
at another student's code or allow your code to be seen or copied. You should
be aware that all code submitted may be subject to automated checks by
plagiarism-detection software.  You should read and understand the school's
policy on student misconduct, which is available at:
\url{http://www.itee.uq.edu.au/about_ITEE/policies/student-misconduct.html}.


\subsection*{Overview}

The assignment is split into two distinct parts. You are required to implement
both parts in the same file system driver, and they must interact as described
below.

When loaded, your file system driver should register itself with type
\textit{ext3301}.

It may be useful to mount any file systems you create with the \texttt{debug}
mount option while testing. This will cause the module to output extensive
information to the kernel's ring buffer describing what it is doing. You may
add new calls to write debugging information out (using the existing
\textit{ext2} debug calls), but please remove any calls to \texttt{printk()}
before submitting your assignment (if you added any).

\subsection*{Part A --- Immediate files}

In the unmodified \textit{ext2} file system, regular files are stored in data
blocks on a block device (often a hard disk). Pointers to these data blocks live
in the inode structure as single, double and triple indirect pointers. These
pointers take up 60 bytes of space in the inode, but some (if not all) are
often unused if the file is small. For more information on the inode structure,
see the \textit{Documentation/filesystems/ext2.txt} file inside the kernel
source tree.

Immediate files are a way of storing the contents of a small file directly in
the inode structure, in the unused block pointers, rather than in data blocks.
Files up to 60 bytes can have their contents stored in these pointers, and no
data blocks need to be allocated. Files over 60 bytes cannot fit in the inode,
so they need to be stored in allocated blocks, and the block pointers need to
be used to point to these data blocks (just like in the existing \textit{ext2}
file system).

Your task for this part of the assignment is to implement immediate files in
the \textit{ext3301} file system. New files should be created as immediate
files (which have the file type value as specified below), and continue to be immediate
files until their contents can no longer fit in the inode itself (by growing to
over 60 bytes). When this happens, the file should be transformed into a
regular file and its contents be transferred to data blocks (you will need to
allocate these before transferring). You need to ensure that when this happens
the block pointers are updated to be valid, so further references to the file
succeed in accessing the data blocks.

When a file is truncated below 60 bytes, you must convert the file back into an
immediate file by transferring the contents of its data blocks into the inode
structure. Ensure you free any data blocks used.

You do not have to implement immediate files for special file types (e.g. block
device) --- you will not be tested on this.

You cannot modify any code outside of the kernel, and since file types are
declared in the system header file \texttt{linux/fs.h} (search for \texttt{DT\_REG}),
you must define a new immediate file type inside your module. Pick a new file type
number, and define the type in the form of:

\texttt{\#define DT\_IM X}

(where \verb|X| is a unique integer. It is not as simple as picking an unused number;
you will need to do some testing to see what works.)

Ensure that \texttt{DT\_IM} is defined as specified in one of your source files
so the marker can recognise an immediate file during marking. It will be
automatically extraced so you may be penalised if the automated script cannot
find it.

Note: since we will be modifying parts of how the inode structure is
represented, the \texttt{e2fsck} tool will probably fail or detect errors when
there aren't any.  This is expected behaviour and you do not need to worry
about fixing this.

Your file system implementation must still be able to mount existing
\textit{ext2} file systems created with the \texttt{mkfs.ext2} tool, so make
sure you do not modify the inode structure itself (do not add or remove any
fields, or change the size of the inode).

\subsubsection*{Tips}

\begin{compactitem}
    \item You can cast the member of the inode structure that stores block
    pointers to an \texttt{unsigned char *} and use it as a contiguous piece of
    memory for storing immediate files in. This way you do not have to access the
    pointer fields directly.
    \item When a file is marked as immediate, you need to ensure no part of the
    file system attempts to access the block pointers (since they will effectively
    contain garbage).
\end{compactitem}

\subsection*{Part B --- File encryption}

The \textit{ext3301} file system must support a very naive encryption scheme.
Any files under the \path{/encrypt} directory (if it exists) of a given
\textit{ext3301} file system need to be encrypted. This will either occur on
disk (if the file is a regular file), or in the inode structure if the file is
marked as immediate. Files outside \path{/encrypt} are to remain as plaintext.

Your file system driver must support an extra mount option called \texttt{key}.
This specifies the encryption key that will be used during encryption and 
decryption for the given mount of the file system. The key will be given in hex
format, and you can use \texttt{sscanf()} to parse the option and extract the
key.  You should only use the 8 least significant bits of the key (meaning it
can only be in the range \texttt{0x0} to \texttt{0xFF}). If no key option is
specified at mount time, then the encryption key defaults to \texttt{0x0}
(which disables encryption) and data should be passed through the file system
as-is.

It is valid to mount a file system with one key, write some data to it, and
then remount it with a different key (even though this would result in garbage
when reading).

Moving a file into the \path{/encrypt} directory should trigger its contents to
be encrypted, and moving a file outside of this directory should trigger its
contents to be decrypted (with whatever key the file system is mounted with).
In Linux, moving a file within the same file system is implemented using a
link and then an unlink on the inode --- the data blocks are not moved. 

You do not have to handle hard links or symbolic links in this file system.

The encryption algorithm is a simple substitution cipher, where each byte is
XORed against the key:

Encryption: $C_i = P_i \oplus k$

Decryption: $P_i = C_i \oplus k$

(where $k$ is the encryption key given when mounting the file system.) Note that
this is a different encryption algorithm from assignment 2.

You should ensure the key does not get written to disk --- it must stay in memory
at all times (so do not store it in the inode).

\subsection*{Interactions between immediate files and encryption}

Your file system needs to support the different combinations of both immediate
files and file encryption. For instance it is perfectly valid to have an
immediate file that is encrypted (and the encrypted data needs to be stored in
the inode as described above).

\subsection*{Short-response questions}

\begin{enumerate}
    \item Discuss what happens in your implementation if you attempt to create a symbolic link from:
    \begin{compactenum}
        \item outside the encrypted directory to a file inside the directory (for instance \path{/foo} $\rightarrow$ \path{/encrypt/foobar})
        \item inside the encrypted directory to a file outside it (for instance \path{/encrypt/foobar} $\rightarrow$ \path{/foo})
    \end{compactenum}
    Explain why each of these either works or does not work, and explain why.

    \item Discuss what happens in your implementation if you attempt to create a hard link from:
    \begin{compactenum}
        \item outside the encrypted directory to a file inside the directory (for instance \path{/foo} $\rightarrow$ \path{/encrypt/foobar})
        \item inside the encrypted directory to a file outside it (for instance \path{/encrypt/foobar} $\rightarrow$ \path{/foo})
        \item another file system to a file inside the encrypted directory
    \end{compactenum}
    Explain why each of these either works or does not work, and explain why.

    \item What would happen if you run \texttt{e2fsck} over an \textit{ext3301} file system that:
    \begin{compactenum}
        \item contains only regular files
        \item contains only encrypted files
        \item contains only immediate files
    \end{compactenum}
    Discuss the behaviour you see and suggest why this happens.
\end{enumerate}

\subsection*{Code compilation}

Your implementation must compile as a Linux kernel module (with a \texttt{.ko}
extension). It must compile and be loadable on the version of the virtual image
provided on the website. See the kernel module practical for information on
Makefiles for kernel modules.

Typing \texttt{make} in your \path{a3} repository directory should produce a
kernel module called \path{a3.ko}.

\subsection*{Coding style}

Your solution must conform to the Linux kernel coding style document available
at \url{http://www.kernel.org/doc/Documentation/CodingStyle} (or
\texttt{Documentation/CodingStyle} in the kernel source tree). You may wish to
run your code through the \texttt{checkpatch} tool to validate that your
solution adheres to the coding style. The following arguments may be useful to
you (where \texttt{FILE} is the C source file you wish to check):

\begin{verbatim}
checkpatch --no-tree --no-signoff -f --no-summary FILE
\end{verbatim}

The tool is available to download from
\url{http://kernel.org/pub/linux/kernel/people/apw/checkpatch/checkpatch.pl-0.29}.

You may also find the \texttt{indent(1)} tool useful with the following arguments:

\begin{verbatim}
indent -kr -i8 FILE
\end{verbatim}

Please note that these are tools only --- they should not be considered
perfect. Always double-check the results by hand to be sure.


\subsection*{Group work}

If you decide to work in groups of two, \textbf{one} member of the group should
e-mail the tutor (or if unavailable the lecturer) \textbf{no later than seven
days before the due date}.  After this cutoff it will be assumed that you are
working individually and you will be marked as such. Both members of a
group will receive the same mark, and any complaints or problems should be
directed to the lecturer who will treat each case confidentially.

The tutor and lecturer's e-mail addresses can be found on the course website.


\subsection*{Submission and Version Control}

The due date for this assignment is \textbf{\duedate}. Submission made after
this time will incur a 10\% penalty per day late (weekends are counted as 1
day). Any submissions more than 4 days late will receive 0 marks. No extensions 
will be given without supporting documentation (i.e.  medical certificate or
family emergency)---should such a situation occur you should e-mail the course
coordinator as soon as possible.

This assignment must be submitted through ITEE's Subversion system (you should
already be familiar with this from previous courses). You should ensure all of
your source files (including any \texttt{.h} files if you have created them)
and \texttt{Makefile} are committed to the repository under the following URL
(where \textit{s4123456} is your student number):

\url{https://svn.itee.uq.edu.au/repo/comp3301-s4123456/\assignnum}

Submissions will be retrieved from your repository when the due date has been
reached.  Your submission time will be taken as the most recent revision in the 
above repository directory.

You are required to make regular commits to your repository as a demonstration
of your work. Subversion history will be considered in marking.

For information regarding ITEE's Subversion system, see
\url{http://studenthelp.itee.uq.edu.au/faq/subversion/}.

Answers to the short-response questions should be provided in a \texttt{responses.txt}
file in the specified repository directory.


\clearpage
\subsection*{Assessment Criteria}

Marks will be awarded based on the modified scheduling behaviour and the coding
style used, as well as your responses to the questions above.

\subsubsection*{Functionality (70 marks)}

\begin{longtable}{p{13cm} r}
    \toprule
    \textbf{Criteria} & \textbf{Marks available} \\

    \midrule
    \textbf{Part A --- Immediate files} & \textbf{40 marks} \\
    New files are created as immediate files with the correct file type & 2 \\
    Immediate files are able to store between 0 and 60 bytes of data in the inode & 7 \\
    Writing less than 60 bytes to an immediate file in user-space stores it in the inode & 3 \\
    Reading to an immediate file in user-space yields the correct data from the inode & 3 \\
    Writing more than 60 bytes to an immediate file converts it to a regular file and transfers the data to disk & 10 \\
    Truncating a regular file to under 60 bytes converts it to an immediate file and transfers the data to inode & 10 \\
    Reading from an existing regular file in user-space yields the correct data & 2 \\
    Writing to an existing regular file in user-space succeeds & 1 \\
    Mounting a \textit{ext2} file system image (with only regular files) succeeds & 1 \\
    No other file types are modified (\textit{e.g.} directory, block device) & 1 \\

    & \\
    \midrule
    \textbf{Part B --- File encryption} & \textbf{20 marks} \\
    \texttt{key} mount option is supported and sets in-memory key for the current mount only & 3 \\
    Writing to a file inside \path{/encrypt} encrypts the contents with the given non-0 key & 4 \\
    Reading from a file inside \path{/encrypt} decrypts the contents with the given non-0 key & 4 \\
    Reading from a file inside \path{/encrypt} when mounted without the \texttt{key} option yields the correct ciphertext & 4 \\
    Moving a file into \path{/encrypt} encrypts its contents with the given non-0 key & 2 \\
    Moving a file out of \path{/encrypt} decrypts its contents with the given non-0 key & 2 \\
    Writing to a file outside of \path{/encrypt} does not modify its inode structure or data blocks & 1 \\

    & \\
    \midrule
    \textbf{Part C --- Combination of immediate and encryption} & \textbf{10 marks} \\
    \multicolumn{2}{l}{\textit{Note that marks in this section are dependant on functionality in Parts A and B}} \\
    A new immediate file can be written to under \path{/encrypt} and have its encrypted contents stored in the inode & 2 \\
    An immediate, encrypted file can be read from user-space successfully & 2 \\
    An immediate, encrypted file can be grown to over 60 bytes and be converted to a regular file & 2 \\
    A regular, encrypted file can be truncated to under 60 bytes and be converted to an immediate file & 2 \\
    An immediate, unencrypted file can be moved into \path{/encrypt} and have its contents encrypted & 1 \\
    An immediate, encrypted file can be moved out of \path{/encrypt} and have its contents decrypted & 1 \\

    \bottomrule
\end{longtable}

\subsubsection*{Documentation, style and build system (10 marks)}

\begin{longtable}{p{13cm} r}
    \toprule
    \textbf{Criteria} & \textbf{Marks available} \\

    \midrule
    Program builds cleanly with no compiler warnings (1 mark will be
        deducted per warning up to a maximum of 3) & 3 \\
    Comments throughout code where appropriate for the marker to understand & 2 \\
    Coding style follows the Linux kernel coding style document (1 mark will be
        deducted per violation up to a maximum of 3) & 3 \\
    Evidence of regular progress through Subversion commit history & 2 \\

    \bottomrule
\end{longtable}

\subsubsection*{Short-response questions (20 marks)}

Your responses will be assessed for their clarity and completeness, and marks
will be assigned as follows:

\begin{longtable}{p{13cm} r}
    \toprule
    \textbf{Question} & \textbf{Marks available} \\
    \midrule

    \textbf{1 --- Symbolic links on encrypted file system} & \textbf{4 marks} \\
    1a: \texttt{/foo} $\rightarrow$ \texttt{/encrypt/foobar} & 2 \\
    1b: \texttt{/encrypt/foobar} $\rightarrow$ \texttt{/foo} & 2 \\

    & \\
    \textbf{2 --- Hard links on encrypted file system} & \textbf{7 marks} \\
    2a: \path{/foo} $\rightarrow$ \path{/encrypt/foobar} & 2 \\
    2b: \path{/encrypt/foobar} $\rightarrow$ \path{/foo} & 2 \\
    2c: from another file system to \path{/encrypt/foobar} & 3 \\

    & \\
    \textbf{3 --- \texttt{e2fsck} on a \textit{ext3301} file system} & \textbf{9 marks} \\
    3a: containing only regular files & 2 \\
    3b: containing only encrypted files & 3 \\
    3c: containing only immediate files & 4 \\

    \bottomrule
\end{longtable}

\end{document}
